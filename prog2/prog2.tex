\documentclass[11pt,a4paper]{article}
\usepackage[utf8]{inputenc}
\usepackage{amsmath}
\usepackage{amsfonts}
\usepackage{amssymb}
\usepackage[margin=1in]{geometry}
\author{Noah Brubaker}
\title{\texttt{dsh:} A Diagnostic Shell}
\newcommand{\nl}{\vspace{10pt}}
\begin{document}

\maketitle
\center{ Course: CSC456 -- Operating Systems }
\center{ Instructor: Dr. Jeff McGough }
\flushleft

\subsection*{Program Description}

This program is a simple diagnostic shell that will emulate some of the functionality of the standard Bash shell. The main purpose of this shell is process identification, and to provide a platform for further development.\nl

The program will provide the following features:
\begin{itemize}

\item \textbf{Prompt:} The prompt \texttt{dsh> } will be displayed. This is where the user will enter commands. 

\item \textbf{Intrinsic Commands:} Six shell intrinsic commands will be implemented: \texttt{cmdnm, signal, systat, exit, cd, and pwd}. \texttt{cmdnm} prints the command that initiated a process. \texttt{signal} sends a signal to another process. \texttt{systat} displays some information about the system, including version, uptime, memory usage, and CPU info. \texttt{exit} will exit the shell nicely. \texttt{cd} implements the \texttt{chdir} command to change the directory via the relative or absolute path provided. \texttt{pwd} prints the working directory.

\item \textbf{Single Program Command:} Any single command (plus arguments), will be executed by the shell and return any stdout.

\end{itemize}

\subsection*{Submission Details}

The submission includes a tar-ball, \texttt{prog1.tgz}, which contains all files relevant to the program. This includes the source code, makefile, and documentation.\nl

\texttt{prog1.tgz} contains: 
\begin{itemize}

\item \texttt{dsh.c}: This file implements the command prompt, command line input, input parsing, and the main event loop for the program.

\item \texttt{run.c} This file implements the intrinsic commands, as well as \texttt{fork/exec} for single commands with arguments.

\item \texttt{makefile} This file builds the program \texttt{dsh} from the source files \texttt{dsh.c} and \texttt{run.c}.

\item \texttt{prog1.pdf} This file provides documentation for the program \texttt{dsh}, its source files and the makefile.

\end{itemize}

\subsection*{Compilation and Usage}

The makefile builds the program in the following way:

\begin{description}
\item \texttt{ gcc dsh.c -o dsh -g -Wall }
\end{description}

The program can be run by typing \texttt{dsh} is a bash shell.

\subsection*{Libraries}

The source code includes the following libraries.

\begin{itemize}
\item \texttt{stdio.h}
\item \texttt{string.h}
\item \texttt{stdlib.h}
\item \texttt{signal.h}
\item \texttt{sys/stat.h}
\item \texttt{sys/wait.h}
\item \texttt{sys/time.h}
\item \texttt{sys/resource.h}
\item \texttt{sys/types.h}
\item \texttt{unistd.h}
\end{itemize}

\subsection*{Structure and Functions}

The general flow of the program has the following format.
\begin{description}
\item \textbf{Program structure}

\texttt{do}\\
~~\texttt{getInput}\\
~~\texttt{parseInput}\\
~~\texttt{status $=$ handler(input)}\\
\texttt{while status $\equiv$ 0}

\end{description}

\subsubsection*{Function Descriptions}

This section describes all functions implemented in the source code.\\

%%%%%%%%%%%%%%%%%%%%%%%%%%%%%%%%%%%%%%%%%%%%%%%%%%%%%%
\begin{description}
\item \textbf{Name:} 
\verb|dsh_prompt[dsh.c(33)]|

\item \textbf{Description:}\\
This function recieves command line input for the shell.
The storage is dynamically allocated for the input stream in
blocks of 256 bytes

\item \textbf{Output:}
\begin{description}
\item \verb|char** input|~~ A pointer to a character array which will store the input taken at the prompt.
\end{description}

\item \textbf{Returns:}
\begin{description}
\item \verb|int -1| ~~ Failed to allocate memory for input
\item \verb|int 0| ~~ Function successful took input
\item \verb|int 1| ~~ No input received on commandline
\item \verb|int 2| ~~ Exit command received
\end{description}
\end{description}\hrule

%%%%%%%%%%%%%%%%%%%%%%%%%%%%%%%%%%%%%%%%%%%%%%%%%%%%%%
\begin{description}
\item \textbf{Name:} 
\verb|parse_input[dsh.c(105)]|

\item \textbf{Description:}\\
This function parses input gathered from the command line.

\item \textbf{Input:}
\begin{description}
\item \verb|char * input| ~~ The input string returned by prompt.
\end{description}

\item \textbf{Output:}
\begin{description}
\item \verb|char *** argv| ~~ A pointer to the new parsed argument list, passed by reference.
\end{description}

\item \textbf{Returns:}
\begin{description}
\item \verb|int argc| ~~ The number of arguments in the input string.
\end{description}
\end{description}\hrule

%%%%%%%%%%%%%%%%%%%%%%%%%%%%%%%%%%%%%%%%%%%%%%%%%%%%%%
\begin{description}
\item \textbf{Name:} 
\verb|run_command[dsh.c(215)]|

\item \textbf{Description:}\\
This function takes the argument list from Main and directs it to either the
 fork/exec code for single functions or to the instrinsic commands.

The first argument is expected to be the command name. 

\item \textbf{Input:}
\begin{description}
\item \verb|int args|~~ Number of arguments
\item \verb|char ** arg_list| ~~ List of arguments
\end{description}

\item \textbf{Returns:}
\begin{description}
\item \verb|int ret|~~ Returns the valued returned by \verb|Run| or \verb|New_Process|.
\end{description}
\end{description}\hrule

%%%%%%%%%%%%%%%%%%%%%%%%%%%%%%%%%%%%%%%%%%%%%%%%%%%%%
\begin{description}
\item \textbf{Name:} 
\verb|main[dsh.c(243)]|

\item \textbf{Description:}\\
This function implements the main event loop for the shell. It waits for the exit command to terminate.

\item \textbf{Returns:}
\begin{description}
\item \verb|int 0|~~ Always returns 0.
\end{description}
\end{description}\hrule

\pagebreak
%%%%%%%%%%%%%%%%%%%%%%%%%%%%%%%%%%%%%%%%%%%%%%%%%%%%%
\begin{description}
\item \textbf{Name:} 
\verb|cmdnm[run.c(30)]|

\item \textbf{Description:}\\
This function gets the command that started a process by accessing \verb|/proc/<pid>/comm|.

\item \textbf{Input:}
\begin{description}
\item \verb|char * pid| ~~ A character array holding the process identification number.
\end{description}

\item \textbf{Returns:}
\begin{description}
\item \verb|int 0|~~ Successful.
\item \verb|int -1| ~~ Couldn't find process.
\end{description}
\end{description}\hrule

%%%%%%%%%%%%%%%%%%%%%%%%%%%%%%%%%%%%%%%%%%%%%%%%%%%%%
\begin{description}
\item \textbf{Name:} 
\verb|send_signal[run.c(63)]|

\item \textbf{Description:}\\
This function sends a signal to a process using the kill command. It checks if the arguments are in the proper ranges, switching them if not.

\item \textbf{Input:}
\begin{description}
\item \verb|char * sig_no|~~ A character array holding the desired signal number.
\item \verb|char * process_id| ~~ A character array holding the process identification number.
\end{description}

\item \textbf{Returns:}
\begin{description}
\item \verb|0|  ~~ Successful.
\item \verb|-1| ~~ Failed to send signal to process.
\end{description}
\end{description}\hrule

%%%%%%%%%%%%%%%%%%%%%%%%%%%%%%%%%%%%%%%%%%%%%%%%%%%%%
\begin{description}
\item \textbf{Name:} 
\verb|systat[run.c(93)]|

\item \textbf{Description:}\\
This function gets some information about the system and displays it for the
user in stdout. The specific information it provides is as follows:\\
~~-Linux version and system uptime\\
~~-Memory Usage: memtotal and memfree\\
~~-CPU Information: vendor id through cache size

\item \textbf{Returns:}
\begin{description}
\item \verb|int 0|~~ Successful.
\item \verb|int neg| ~~ Couldn't access directory.
\end{description}
\end{description}\hrule

%%%%%%%%%%%%%%%%%%%%%%%%%%%%%%%%%%%%%%%%%%%%%%%%%%%%%
\begin{description}
\item \textbf{Name:} 
\verb|cd[run.c(176)]|

\item \textbf{Description:}\\
This function implements the change directory intrinsic command.

\item \textbf{Input:}
\begin{description}
\item \verb|char * path|~~ The absolute or relative path to the desired directory.
\end{description}

\item \textbf{Returns:}
\begin{description}
\item \verb|0| ~~ Successful.
\item \verb|-1| ~~ No such file or directory.
\end{description}
\end{description}\hrule

%%%%%%%%%%%%%%%%%%%%%%%%%%%%%%%%%%%%%%%%%%%%%%%%%%%%%
\begin{description}
\item \textbf{Name:} 
\verb|pwd[run.c(203)]|

\item \textbf{Description:}\\
This function implements the print working directory intrinsic command.

\item \textbf{Returns:}
\begin{description}
\item \verb|0|~~ Always returns 0.
\end{description}
\end{description}\hrule

%%%%%%%%%%%%%%%%%%%%%%%%%%%%%%%%%%%%%%%%%%%%%%%%%%%%
\begin{description}
\item \textbf{Name:} 
\verb|Run[run.c(221)]|

\item \textbf{Description:}\\
 This function directs the program to run the intrinsic commands, checking for correct number of arguments where applicable.

\item \textbf{Input:}
\begin{description}
\item \verb|int cmd_num|~~ Number specifying desired command.
\item \verb|int args| ~~ The number of arguments.
\item \verb|char ** arg_list| ~~ The null-terminated list of arguments.
\end{description}

\item \textbf{Returns:}
\begin{description}
\item \verb|int ret|~~ The return value of function it calls
\item \verb|int neg|~~ Wrong number of inputs or similar error.
\item \verb|int 2| ~~ Exit code.
\end{description}
\end{description}\hrule

%%%%%%%%%%%%%%%%%%%%%%%%%%%%%%%%%%%%%%%%%%%%%%%%%%%%%
\begin{description}
\item \textbf{Name:} 
\verb|New_Process[run.c(269)]|

\item \textbf{Description:}\\
Creates a new process to run the given single command received at the command line in the diagnostic shell.

\item \textbf{Input:}
\begin{description}
\item \verb|char ** arg_list|~~ The list of arguments for the given command.
\end{description}

\item \textbf{Returns:}
\begin{description}
\item \verb|int 0|~~ If fork and exec operations were successful.
\item \verb|int -1| ~~ An error occured. Either couldn't find command or failed to execute it.
\end{description}
\end{description}\hrule

%%%%%%%%%%%%%%%%%%%%%%%%%%%%%%%%%%%%%%%%%%%%%%%%%%%%%%%

\subsection*{Testing and Verification}

This program was tested and verified by trying each required command at separate times. The code was developed in such a way that functionally was continually added to an already functional program. Since each required feature was largely independent of the others, debugging was straight-forward.\nl

Valgrind was used to check for memory leaks. There are no known bugs at the time of submission, however error checking could be a bit more rigorous. 

\end{document}
